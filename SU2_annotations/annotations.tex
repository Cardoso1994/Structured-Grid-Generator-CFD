\documentclass[letterpaper, openright, 12pt]{book}
\usepackage[spanish]{babel}
\usepackage[utf8]{inputenc}

\usepackage{graphicx} % para imagenes
\usepackage{subfigure} % para subfiguras
\usepackage{caption}

\usepackage{physics} % para formulas matemáticas
\usepackage{amsmath}

\usepackage[left=2cm,top=2cm,right=2cm,bottom=2cm]{geometry} % controla márgenes
\usepackage{cite} % para dar formato a las referencias bibliográficas
\usepackage{enumerate} % para hacer listas
\usepackage[hidelinks]{hyperref}
\usepackage{cleveref}



% Cabiar el titulo por default de los indices
\addto{\captionsspanish}{\renewcommand*{\listfigurename}{Indice de Figuras}} %cambia el nombre del indice de figuras
\addto{\captionsspanish}{\renewcommand*{\contentsname}{Indice}}
\addto{\captionsspanish}{\renewcommand*{\listtablename}{Indice de Tablas}}


\pagestyle{plain} %sin cabeceras en las páginas, numero de pagina centrado abajo
\begin{document}
    \chapter{Métodos Implícitos y Explícitos}
        Estos métodos se utilizan para la solución
        de ecuaciones diferenciales ordinarias y
        parciales, para problemas dependientes del
        tiempo.
    \section{Métodos Implícitos}
        Los métodos implícitos llevan a cabo la
        solución resolviendo el sistema de
        ecuaciones simultaneamente para el tiempo
        actual y un instante de tiempo posterior.
    \section{Métodos Explícitos}
        Los métodos explícitos calculan los
        resultados de un tiempo posterior con base
        en los resultados obtenidos en el momento
        actual.

\end{document}
